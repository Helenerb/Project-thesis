\documentclass[11pt,a4paper,twoside]{article}
\usepackage[utf8]{inputenc}
\usepackage{stylesheet}
\linespread{1.3}

\usepackage{import}
\usepackage{graphicx}
\usepackage{amsmath}
\usepackage{bm}

\numberwithin{equation}{section}
\numberwithin{figure}{section}
\usepackage{natbib}
%\usepackage{biblatex}
%\usepackage[round]{natbib}  % formatting references as Name (Year) and (Name, Year)
% use parencite for all citations 
\renewcommand{\cite}[1]{\parencite{#1}}
\usepackage{caption}
\usepackage{subcaption}
\usepackage{appendix}
%\usepackage[table,xcdraw]{xcolor}
\usepackage{blindtext}
\usepackage{url}

%try better bibliography formatting
\usepackage[style=authoryear, backend=biber]{biblatex}
\addbibresource{bibliography.bib}
\usepackage[nottoc]{tocbibind}
\setlength{\bibitemsep}{.5\baselineskip}

\captionsetup[table]{justification=centering,singlelinecheck=false} % table caption is left-aligned
%\bibliographystyle{plainnat} % reference list style

%\setlength\parindent{0pt} % remove indentation
%\setlength{\parskip}{1em} % add space between columns

%\import{}{Acronyms.tex}


\begin{document}

\maketitle\thispagestyle{empty}
\clearpage\pagenumbering{roman}

\section*{Abstract}
Mortality forecasts are important in e.g. demographical and actuarial sciences, and the Lee-Carter model \parencite{LeeCarter1992} is widely used for this purpose. Until now, it has not been possible to use the popular \inla methodology to perform Bayesian inference with this model. We investigate if the method proposed by \textcite{BachlLindgren2019}, implemented in the \texttt{R} library \inlabru, enables Bayesian inference with the \inla methodology on two versions of the Lee-Carter model for mortality projections.
Our research consists of two phases. In the first phase, we test the aforementioned hypothesis for synthetically produced data. In the second phase, we test the hypothesis on real data for cancer mortality, and we use the method of \textcite{BachlLindgren2019} for both estimation and prediction of cancer mortality rates. The results from both parts of the study indicate that the method of \textcite{BachlLindgren2019} is a suitable tool for performing Bayesian inference on the Lee-Carter models. We compare the prediction results from the investigation on real data using the Lee-Carter models to prediction results obtained using Age-Period-Cohort (APC) models. 

% Table of contents
\clearpage\tableofcontents
\newpage
\printacronyms

% Main body
\clearpage\pagenumbering{arabic}
\import{}{Report/1-Introduction.tex}
\import{}{Report/2-Theory.tex}
\import{}{Report/3-Synthetic-data.tex}
\import{}{Report/4-Real-data.tex}


% Reference list
\clearpage
\section*{References}
\addcontentsline{toc}{section}{References}
\printbibliography[heading=none]
%\bibliographystyle{apalike}
%\bibliographystyle{plainnat}
%\bibliography{Bibliography}

% Appendix
\begin{appendices}
\clearpage
\import{}{Report/Appendix.tex}
\end{appendices}

\end{document}