\section{Introduction}

% \newpar \textcolor{myBluePurple}{Par 1, v1: }Good statistical methods for estimation and forecasting of changes in population over time is necessary in many areas. The ability to forecast changes in population, such as mortality, fertility, immigration and emigration can be a useful tool in e.g. actuarial or demographical sciences. Specifically, the ability to forecast changes in mortality rates in a good way is an important tool in for example medical and pharmaceutical research, planning of health policies, correct pricing of medical or life insurances
Good statistical methods for forecasts of population change, and specifically of changes in mortality rates, are important tools in several scientific fields. In the public sector, mortality projections are of key interest in, among others, the development of health policies and planning of social security services and pension funds, and in actuarial applications, mortality forecasts plays an important role in the development of life insurances (\textcite{BROUHNS2002373}, \textcite{RENSHAW2006556}, \textcite{CZADO2005260}, \textcite{LeeCarter1992}). 

\newpar A common way to model population development is to divide the population into age groups, for each period in time. This can be arranged in a so-called Lexis-diagram \parencite{CZADO2005260}, where age groups lie along the vertical axis and calendar time periods lie along the horizontal axis. The age groups and periods are parted into intervals, so that the diagram consists of squares of equal size, representing the part of the population that is in a given age group at a given period in time. Squares that lie on the same diagonal correspond to the same year of birth, or the same "cohort". Methods that are based on this representation of the population usually assume constant mortality rates within these units \parencite{CZADO2005260}, typically with some correlation or smoothing over the Lexis plane (\textcite{CZADO2005260}, \textcite{RieblerThesis2010}). 

\newpar Two models that describe population development by these principles are the Lee-Carter model \parencite{LeeCarter1992} and the Age-Period-Cohort (APC) model \parencite{Clayton1987}. Throughout this paper, we consider these models in the scope of forecasting population development due to changes in mortality rates, as this seems to be their main field of application in other literature (\textcite{LeeCarter1992}, \textcite{RieblerThesis2010}, \textcite{CZADO2005260}, \textcite{BROUHNS2002373}, \textcite{RENSHAW2006556}). However, as \textcite{Wisniowski2015} argue, these models could also be used with other types of population change, such as fertility, immigration and emigration, which \textcite{Wisniowski2015} also do with convincing results. 

\newpar The Lee-Carter model was originally proposed by \textcite{LeeCarter1992} to forecast development in the US population, and has since been widely used to model both cause-specific and general mortality \parencite{GirosiKing2007}. After its introduction, numerous extensions of the model have been proposed. \textcite{BROUHNS2002373} introduced the extension of the Lee-Carter model with a Poisson distribution, and \textcite{CZADO2005260} extended this further by incorporating the model in a Bayesian framework. Adding a cohort effect to the Lee-Carter model was suggested by \textcite{RENSHAW2006556}, and this has been successfully applied by e.g. \textcite{Wisniowski2015}.

\newpar The APC model \parencite{Clayton1987} is another widely used model for mortality projections. The model was proposed to model cause-specific mortality \parencite{Clayton1987}, unlike the Lee-Carter model that was proposed to model general mortality on larger populations \parencite{LeeCarter1992}, and this seems to be a common field of application for APC models (see e.g. \textcite{rieblerHeld2010}, \textcite{RieblerHeldRue2012}, \textcite{rieblerSwissSuicide2012}). The APC model is endowed with a well-known identifiability problem, which means that the contributions from the different effects of time (age, period and cohort) are not distinguishable from each other in mortality projections produced with this model \parencite{RieblerThesis2010, RieblerHeldRue2012}. As this identifiability issue is thoroughly researched in other literature (we refer to e.g. \textcite{RieblerThesis2010}), we do not investigate it further here. 

\newpar We consider these models for population development in a Bayesian framework. An advantage of doing so is that it, as argued by \textcite{Wisniowski2015}, provides a natural framework for for modelling of uncertainty through probability distributions. The probabilistic uncertainty that naturally follows population forecasts is reflected by the resulting posterior probability distribution. Uncertainty about model parameters and, as also mentioned by \textcite{Wisniowski2015}, expert knowledge with its corresponding uncertainty, can be included through prior probability distributions.

The integrated nested Laplace approximation (INLA) \parencite{rue2009inla} is one method to perform Bayesian inference that, due to its computational power, is very popular. The \inla method is applicable to the APC model, and inference with \inla has been successfully done on many occasions (see e.g. \textcite{RieblerThesis2010}, \textcite{RieblerHeldRue2012}, \textcite{rieblerSwissSuicide2012}). 

\newpar For the Lee-Carter model, it is not possible to use the INLA method for inference as it is. This is due to a linearity requirement for models applicable to the INLA method \parencite{martinoRiebler2019}, which the Lee-Carter model does not fulfill. Other literature have performed Bayesian inference on Lee-Carter types of model using typically Markov chain Monte Carlo (MCMC) sampling methods (see \textcite{CZADO2005260}, \textcite{Wisniowski2015}). The disadvantage with this approach is its tediousness, as the MCMC method may require hours or days to obtain the same accuracy on the results as the INLA method can produce in seconds or minutes \parencite{rue2009inla}. Finding a way to use the INLA methodology with the popular Lee-Carter model would therefore be beneficial. 

\newpar Recently, \textcite{BachlLindgren2019} proposed a method, implemented in the \texttt{R} library \inlabru, for applying the INLA methodology to models not fulfilling the linearity requirement, arising in an ecological setting. We investigate if the same method will make it possible to use the INLA methodology also on Lee-Carter models. If this turns out to be the case, it would open up to e.g. more efficient sensitivity analysis of model parameters and feasible comparison of different model choices for Lee-Carter type of models.

% \newpar \textcolor{myDarkGreen}{Considering these models in a Bayesian setting. Commonly done with the APC-model. Possible to use the fast and powerful method of inference Integrated Nested Laplace Approximation (INLA). Widely used with good effect. Refer to example of Andrea, swiss suicide mortality. }
% \newline
% \newpar \textcolor{myDarkGreen}{Lee-Carter model, with extensions also considered in a Bayesian framework, but not possible to use Inla. Kanskje også si noe om tradisjonell SVD + ARIMA approach? Hva brukte Lee-Carter originalt? Other literature have used various different methods, \textcite{CZADO2005260} used MCMC with Gibbs sampling and Metropolis Hastings sampling to fit a Poisson-extenstion of the Lee-Carter model to male French population data. \textcite{Wisniowski2015} also worked with a Poisson version of the Lee-Carter model, extended with a cohort effect to model population changes in data for the United Kingdom. They used MCMC sampling to perform the Bayesian inference. Problem with these methods is: tedious. Need to run long to get good accuracy compared to INLA method.}
% \newline 
% \newpar \textcolor{myDarkGreen}{Inlabru tool: proposed by \textcite{BachlLindgren2019} to solve a similar problem for a different kind of model arising in ecological data. We investigate if applying this method to the Lee-Carter methods will enable fast bayesian inference with inla. This would be an advantage because: inla produce fast and accurate results --> possibility for efficient sensitivity analysis and comparison of many different model choices in an efficient way. Lee-Carter models are widely used --> good to have efficient inference method.} 
