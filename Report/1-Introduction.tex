\section{Introduction}
\textcolor{myDarkBlue}{Unfinished}
\textcolor{myDarkGreen}{
Population forecasting: why is it important? 
Insurance
Demographical sciences}

\newpar \textcolor{myBluePurple}{Par 1, v1: }Good statistical methods for estimation and forecasting of changes in population over time is necessary in many areas. The ability to forecast changes in population, such as mortality, fertility, immigration and emigration can be a useful tool in e.g. actuarial or demographical sciences. Specifically, the ability to forecast changes in mortality rates in a good way is an important tool in for example medical and pharmaceutical research, planning of health policies, correct pricing of medical or life insurances
\newline
\textcolor{myBluePurple}{Par 1, v2: }Good statistical methods for forecasts of population change, and specifically of changes in mortality rates, are important tools in several areas. In the public sector, mortality projections are of key interest in, among others, the development of health policies and planning of social security services and pension funds, and in actuarial applications, mortality forecasts plays an important role in the development of life insurances (\textcite{BROUHNS2002373}, \textcite{RENSHAW2006556}, \textcite{CZADO2005260}, \textcite{LeeCarter1992}). 

\newline
\newpar \textcolor{myDarkGreen}{Lexis diagram: A common way to model population development is to part the population by age and period in time. This can be arranged in a so-called Lexis-diagram, originally proposed by cite Lexis. Displayed in Figure..., illustrated with age groups and periods of one year. Each square corresponds to the part of the population in a given age group during a given period. Squares that lie on the same diagonal correspond to the same year of birth, or the same "cohort". Two models that utilise this principle are the Lee-Carter and the APC model. Model population change as function of age, period and cohort effects.}
\newline \newline
A common way to model population development is to part the population by age and period in time. This can be arranged in a so-called Lexis-diagram, named after its inventor \parencite{CZADO2005260}, where age groups lies along the y-axis and calendar time periods lie along the x-axis. The age groups and periods are parted into intervals, so that the diagram consists of units of equal size, representing the part of the population in a given age group at a given period in time. Squares that lie on the same diagonal correspond to the same year of birth, or the same "cohort". Methods that are based on this populational representation usually assume constant mortality rates within these units \parencite{CZADO2005260}, usually with some correlation or smoothing over the Lexis plane (\textcite{CZADO2005260}, \textcite{RieblerThesis2010}). 
\newline
Two models that utilise this principle are the Lee-Carter and the APC model. Throughout this paper, we mainly consider these models in the scope of forecasting population development due to changes in mortality rates, as this also seem to be their main field of application in other literature (\textcite{LeeCarter1992}, \textcite{RieblerThesis2010}, \textcite{CZADO2005260}, \textcite{BROUHNS2002373}, \textcite{RENSHAW2006556}). However, as \textcite{Wisniowski2015} argues, these principles could also be used with other types of population change, such as fertility, immigration and emigration, which they also do with convincing success. 
\newline
\newpar An advantage of considering population estimates and forecasts in a Bayesian framework is that it, as argued by \textcite{Wisniowski2015}, provides a natural framework for for modelling of uncertainty through probability distributions. The probabilistic uncertainty that naturally follow population forecasts is reflected by the resulting posterior probability distribution. Uncertainty about model parameters and, as also mentioned by \textcite{Wisniowski2015}, expert knowledge with its corresponding uncertainty, can be included through prior probability distributions.
\newline
The Lee-Carter model was originally proposed by \textcite{LeeCarter1992} to forecast development in the US population, and has been widely used to model both cause-specific and general population changes \parencite{GirosiKing2007}. After its introduction by \textcite{LeeCarter1992}, numerous extensions of the model have been proposed. \textcite{BROUHNS2002373} introduced the extension of the Lee-Carter model with a Poisson distribution, and \textcite{CZADO2005260} extended this further by incorporating the model in a Bayesian framework. Adding a cohort effect to the Lee-Carter model was suggested by \textcite{RENSHAW2006556}, and this has been successfully applied by e.g. \textcite{Wisniowski2015}.
\newline

\newpar \textcolor{myDarkGreen}{The APC model is a widely used alternative to the Lee-Carter model. Finn ut: hvor er den mest brukt? Hent litt inspirasjon fra Andrea sine papers.} 
\newline
\newpar \textcolor{myDarkGreen}{Considering these models in a Bayesian setting. Commonly done with the APC-model. Possible to use the fast and powerful method of inference Integrated Nested Laplace Approximation (INLA). Widely used with good effect. Refer to example of Andrea, swiss suicide mortality. }
\newline
\newpar \textcolor{myDarkGreen}{Lee-Carter model, with extensions also considered in a Bayesian framework, but not possible to use Inla. Kanskje også si noe om tradisjonell SVD + ARIMA approach? Hva brukte Lee-Carter originalt? Other literature have used various different methods, \textcite{CZADO2005260} used MCMC with Gibbs sampling and Metropolis Hastings sampling to fit a Poisson-extenstion of the Lee-Carter model to male French population data. \textcite{Wisniowski2015} also worked with a Poisson version of the Lee-Carter model, extended with a cohort effect to model population changes in data for the United Kingdom. They used MCMC sampling to perform the Bayesian inference. Problem with these methods is: tedious. Need to run long to get good accuracy compared to INLA method.}
\newline 
\newpar \textcolor{myDarkGreen}{Inlabru tool: proposed by \textcite{BachlLindgren2019} to solve a similar problem for a different kind of model arising in ecological data. We investigate if applying this method to the Lee-Carter methods will enable fast bayesian inference with inla. This would be an advantage because: inla produce fast and accurate results --> possibility for efficient sensitivity analysis and comparison of many different model choices in an efficient way. Lee-Carter models are widely used --> good to have efficient inference method.} 
