\subsection{Multivariate Lee-Carter and APC Models}
\label{sec:multivariateAPC}
In many cases, it is useful to model mortality rates for subgroups of a population. We might have data for different geographical areas, where there is reason to believe that the mortality rates are not independent of what region the population belongs to. Or, we might want to obtain male and female mortality rates specifically, without treating the male and female populations as all-together separate. We consider the case of male- and female-specific mortality, and extend the Lee-Carter and APC models to include sex as a grouping factor. There are several ways to extend the models to include correlation between the groups (see e.g. \textcite{Wisniowski2015} and \textcite{RieblerHeldRue2012}). We follow the straightforward approach proposed by \textcite{rieblerHeld2010}, and construct multivariate versions of the Lee-Carter and APC models by assuming that some of the age, period or cohort effects are common for both sexes, while the overall mortality level and remaining effects are specific for each sex. For example, when we assume that the age effects are similar for both sexes, while the period and cohort effects are specific for males and females, the APC model is formulated as
\begin{equation}
    Y_{x,t}^{\text{sex}} \sim \Poisson(E_{x,t}^{\text{sex}}e^{\eta_{x,t}^{\text{sex}}}), \quad \eta_{x,t}^{\text{sex}} = \mu^{\text{sex}} + \rho_x + \phi_t^{\text{sex}} + \psi_{x,t}^{\text{sex}} + \epsilon_{x,t}^{\text{sex}},
\end{equation}
where $Y_{x,t}^{\text{sex}}$ is the observed mortality rate for one sex at age group $x$ and time $t$ and $E_{x,t}^{\text{sex}}$ is the corresponding population at risk. 