\subsection{Multivariate Age, Period, Cohort - models}
\label{sec:multivariateAPC}
\textcolor{myDarkGreen}{
Skriv om hvordan vi bruker APC og LCC som multivariate modeller. Skriv om at man kan holde én, flere eller ingen (ekvivalent til to univariate modeller) felles. Dersom det er korrelasjon mellom de to, vil en felles effekt gi en bedre prediksjon, mens forskjellige (ufelles) effekter vil kunne reflektere kovariat-vise effekter. 
}
In many cases it is useful to include more variables than just age, period and cohorts in a model for mortality prediction. It might be that we have data sets for different geographical regions, where there is reason to believe that the expected mortality rates are not independent of these regions. Another common situation is to have data for male and female, for which some, but not perfect, correlation is expected. Because of this, a common extension of APC and Lee-Carter models are multivariate APC and Lee-Carter models. 

We consider the extension of Lee-Carter and APC models to include sex as a covariate. A simple way to create a multivariate model, is to keep some of the time effects common, and some effects separate, for male and female. In this framework, keeping all effects common corresponds to considering perfect correlation between male and female, and keeping no effects common corresponds to treating male and female as uncorrelated populations. 

\textcolor{myDarkGreen}{This needs to be better, also include some references. }