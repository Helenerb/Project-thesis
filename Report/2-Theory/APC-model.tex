\subsection{The Age-Period-Cohort model}
\label{sec:APC-model}
The Age-Period-Cohort (APC) model is an alternative to the Lee-Carter model, that is also widely used for mortality projections and forecasts \parencite{rieblerHeld2010}. The model considers mortality rate as a function of the age group, period (typically calendar year) and cohort (typically birth year) of the relevant population. The model has a similar structure to the Lee-Carter model, assuming that 
\begin{equation}
    Y_{x,t} \sim \Poisson(E_{x,t}\cdot e^{\eta_{x,t}})
    \label{eq:APClikelihood}
\end{equation}
\parencite{RieblerThesis2010}. Again, $Y_{x,t}$ are the observed mortality rates for age group $x$ at period $t$, and $E_{x,t}$ is the number of people at risk for age group $x$ at period $t$. The expression for the predictor $\eta_{x,t}$ differs from the Lee-Carter model, and is modelled as
\begin{equation}
    \etax = \mu + \rho_x + \phi_t + \psi_k + \epsilon_{x,t}.
    \label{eq:APCmodel}
\end{equation}
Here, $\mu$ is an intercept, $\rho_x$ is an age effect, $\phi_t$ is a period effect, $\psi_k$ is a cohort effect and $\epsilon_{x,t}$ is a random term. Note that in the classical formulation of the APC model, described in e.g. \textcite{Clayton1987}, the random term $\epsilon_{x,t}$ is not included. Inspired by \textcite{RieblerThesis2010} and \textcite{Besag1995}, we add this term to model randomness in mortality rates that cannot be explained by either age, period or cohort effects, and also because this makes the model more easily comparable to the Lee-Carter models, which have a similar term. 
\newline
\noindent To ensure identifiability of the intercept $\mu$ it is common practice to impose sum-to-zero constraints on the random effects \parencite{RieblerThesis2010}:
\begin{equation}
    \sum_{x}\rho_x = 0, \quad \sum_{t}\phi_t = 0, \quad \sum_{k} \psi_k = 0. 
    \label{eq:APCconstraints}
\end{equation}
$\mu$ may be interpreted as an overall mortality level \parencite{RieblerThesis2010} and the effects $\rho_x$, $\phi_t$ and $\gamma_k$ may be interpreted as age- period- and cohort specific variations in the mortality rate, relative this overall level. 
\newpar The APC-model is subject to a well-known identifiability problem, due to the linear structure of the predictor $\eta_{x,t}$ \parencite{RieblerThesis2010}. This implies that while we can find well-defined values for $\mu$ and $\eta_{x,t}$, given the constraints in Expression \ref{eq:APCconstraints}, we cannot uniquely distinguish the age, period and cohort effects from one another. This means that when we consider results from analyses using the APC model, we only look at the value of the predictior $\eta_{x,t}$, and by that the mortality rate, not the individual effects. This problem has been widely discussed in previous literature, so we do not explore it further here and we refer the interested reader to e.g. \textcite{RieblerThesis2010}. 
\newpar As it is reasonable to believe that the effects $\rho_x$, $\phi_t$ and $\psi_k$ are quite smooth (one expects the mortality rates for age-groups, periods and cohorts that are close to each other to be similar), this is usually reflected in the model choices of $\rho_x$, $\phi_t$ and $\psi_k$. \textcite{RieblerThesis2010} discusses using both random walks of order one (RW1), which penalize deviation from a constant, and random walks of order two (RW2), which penalize deviation from a linear trend, to model the time effects.

% \textcolor{myDarkGreen}{Old text: }
% The model uses three effects to analyze population data; age effects, period effects, which is typically related to the calendar year of an observation and cohort effects, which are usually equivalent to the birth-year of the observed subjects. It has been shown that these three effects combined are able to capture large parts of the underlying causes of age- and time-specific causes of mortality in a population. 

% We use the formualtion of the APC model that is described in \textcite{rieblerHeld2010}. 
% They assume that $n_{x,t}$ is the number of persons at risk in age group $x$, $x = 1,\ldots,X$ during period $t$, $t = 1,\ldots,T$, that $y_{x,t}$ is the number of cases in age group $x$ at during period $t$. They further assume that 
% \begin{equation}
%     y_{x,t} \sim \Poisson(n_{x,t}\cdot \lambda_{x,t})
% \end{equation}
% and that the likelihood for the whole data is given by the product of the corresponding Poisson terms [reformulate!!! And understand exactly what this means. ]
% The APC model is then given by structuring the logarithm of $\lambda_{x,t}$ as a linear predictor consisting of age-, period- and cohort-related terms:
% \begin{equation}
%     \etax = \log(\lambda_{x,t}) = \mu + \rho_x + \phi_t + \psi_k.
%     \label{eq:APCmodel}
% \end{equation}
% Here, $\mu$ is the intercept (or overall mortality level), $\rho_x$ is an age effect, $\phi_t$ is a period effect and $\psi_k$ is a cohort effect. The index $k$, $k = 1,\ldots,K$ in $\psi_k$ is a birth cohort index, which depends directly on the age index $x$ and period index $t$. When the age and period intervals are of equal width, this is given by
% \begin{equation*}
%     k = (X - x) + t, \quad K = (X - 1) + T.
% \end{equation*}
% When the intervals are of different width, e.g if the age intervals are of length $M$ and the period is measured yearly, the cohort index is given by
% \begin{equation}
%     k = M \cdot (X - x) + t, \quad K = M \cdot (X - 1) + T.
%     \label{eq:cohortIndex}
% \end{equation}
% To ensure identifiability of the intercept $\mu$, sum-to-zero contstraints are imposed on the remaining effects:
% \begin{equation}
%     \sum_{x = 1}^{X}\rho_x = 0, \quad \sum_{t = 1}^T\phi_t = 0, \quad \sum_{k = 1}^K \psi_k = 0. 
%     \label{eq:APCconstraints}
% \end{equation}