\newpage
\section{Bayesian Inference on Models for Forecasting Population Development}
\label{sec:BayesianInference}
We consider both the Lee-Carter models and the APC model in a Bayesian setting, and investigate how we can use the \inla method to find posterior distributions for age-, period- and cohort-specific mortality rates. 


\subsection{Bayesian Inference on the APC Model}
\label{sec:BayesianInferenceAPC}
When we consider the APC model as a Bayesian model, we treat the age, period and cohort effects as unknown, with a likelihood model given by Expression \ref{eq:APClikelihood}. Our goal is to find a posterior distribution of these effects, given some observed mortality rates $Y_{x,t}$, and thus we need to assign prior distributions to the random effects. Inspired by \parencite{RieblerThesis2010}, we consider two different choices of prior distributions for the random effects. The first is an RW1, given by
\begin{equation}
    \rho_x = \rho_{x-1} + \epsilon_x, \quad \epsilon_x \sim \Normal(0,1/\tau_{\rho1}),
    \label{eq:APCrw1}
\end{equation}
with equivalent expressions for $\phi_t$ and $\psi_k$.
The second is an RW2, which assumes an iid gaussian distribution on the second order differences of the increments in the effects, and is given by
\begin{equation}
    \rho_x = 2\rho_{x-1} - \rho_{x-2} + \epsilon_x, \quad \epsilon_x \sim \Normal(0, 1/\tau_{\rho2}),
    \label{eq:APCrw2}
\end{equation}
with equivalent expressions for $\phi_t$ and $\psi_k$. Following e.g. \textcite{Besag1995} we assign an iid gaussian prior to the effect $\epsilon_{x,t}$:
\begin{equation}
    \epsilon_{x,t} \sim \Normal(0,1/\tau_\epsilon).
\end{equation}
For the hyperparameters $\tau_{\rho 1}, \tau_{\rho 2}, \tau_{\phi 1}, \tau_{\phi 2}, \tau_{\psi 1}, \tau_{\psi 2}, \tau_\epsilon$ we assign a type of prior called Penalized complexity (PC) priors, which was proposed by \textcite{SimpsonRueRiebler2017} to by well-suited for random walk models \parencite{Rubio2020}. The PC priors penalize deviation from a base model, and by that rewarding less complex models. For the precisions $\tau$, the PC-priors are defined by the statement \begin{equation}
    \mathbb{P}(\sigma > \sigma_0) = \alpha,
\end{equation}
where $\sigma = 1/\sqrt{\tau}$ is the standard deviation, and $\sigma_0$ and $\alpha$ are parameters that we may decide. 

From this point, we denote the APC model with RW1 priors (Equation \ref{eq:APCrw1}) as the APC1-model and the APC model with RW2 priors (Equation \ref{eq:APCrw2}) as the APC2-model. As described in Section \ref{sec:InlaRestrictions}, we can easily use the \inla method to perform the inference. 

\subsection{Bayesian Inference with the Lee-Carter Models}
\label{sec:BayesianInferenceLC}
We consider the Lee-Carter model in a Bayesian setting, meaning that we treat the effects $\alpha_x$, $\beta_x$, $\kappa_t$ and $\gamma_k$ as unknown random effects, with a likelihood model given by Equation \ref{eq:LC-model} or \ref{eq:LCC-model-orig} and with some prior distributions. As discussed in Section \ref{section:Lee-Carter}, a random walk with drift is normally used to model the period effect $\kappa_t$, and so this is the natural choice of the prior distribution for $\kappa_t$:
\begin{equation}
    \kappa_t = \kappa_{t-1} + \phi  + \epsilon_t,\quad \epsilon_t \sim \Normal(0,\tau_{\kappa}).
    \label{eq:randomWalkDrift}
\end{equation}
Here, $\phi$ is the drift term and $\tau_{\kappa}$ is the precision in the normal distribution of the random term $\epsilon_t$ of the random walk. We assign driftless random walks as a priors to $\alpha_t$ and $\gamma_k$:
\begin{equation}
    \begin{aligned}
        \alpha_x &= \alpha_{x-1} + \epsilon_{x}, \quad \epsilon_x \sim \Normal(0,1/\tau_{\alpha}) \\
        \gamma_k &= \gamma_{k-1} + \epsilon_{k}, \quad \epsilon_k \sim \Normal(0,1/\tau_{\gamma}), \\
    \end{aligned}
\end{equation}
where $\tau_{\alpha}$ and $\tau_{\gamma}$ are the precisons of the random terms $\epsilon_x$ and $\epsilon_k$, respectively, of the two random walks. Taking into consideration that $\beta_x$ often takes a less smooth shape than the other effects (see Section \ref{section:Lee-Carter}), we assign a Gaussian iid prior to $\beta_x$:
\begin{equation}
    \beta_x \sim \Normal(0, 1/\tau_{\beta}).
\end{equation}
Since $\epsilon_{x,t}$ is included to model randomness in the output, it is natural to choose a gaussian iid prior for this effect as well:
\begin{equation}
    \epsilon_{x,t} &\sim \Normal(0, 1/\tau_{\epsilon}).
\end{equation}
In the two expressions above, $\tau_\beta$ and $\tau_\epsilon$ are the precisions in the respective normal distributions.  
As for the APC model, we assign PC priors to the hyperparameters $\tau_\kappa, \tau_\alpha, \tau_\gamma, \tau_\beta, \tau_\epsilon$.

\subsubsection{Reformulation of the Lee-Carter model}
For computational reasons, we rewrite the original Lee-Carter model slightly. We reformulate $\kappa_t$ as the sum of a linear effect and a drift-less random walk:
\begin{equation}
    \kappa_t = \kappa_t^* + \phi\cdot t, \quad \kappa_t^* = \kappa_{t-1}^* + \epsilon_t, \quad \epsilon_t \sim \Normal(0,\tau_{\kappa}).
\end{equation}
We impose a sum-to-zero constraint on $\kappa_t^*$ (instead of on $\kappa_t$, as previously described). Since $\sum_t \phi \cdot t \neq 0$ for $t > 0$, which will be true for all realistic time indices, we obtain a difference between the rewritten and the original values of the period-effect. We include this shift as an intercept in our model. We also rewrite our model for the age-effect $\alpha_x$ as 
\begin{equation}
    \alpha_x = \bar{\alpha_x} + \alpha_x^*.
\end{equation}
Here $\bar{\alpha_x}$ is the average of $\alpha_x$ for all ages $x$, and we include this term in the intercept as well. We then get an additional sum-to-zero constraint on $\alpha_x^*$. Denoting, from this point, $\alpha_x^*$ by $\alpha_x$ and $\kappa_t^*$ by $\kappa_t$, we get the rewritten from of the Lee-Carter and the cohort-extended Lee-Carter model as:
\begin{equation}
    \eta_{x,t} = \mu + \alpha_x + \beta_x(\phi \cdot t + \kappa_t) + \epsilon_{x,t}
    \label{eq:LC-rewritten}
\end{equation}
and
\begin{equation}
    \eta_{x,t} = \mu + \alpha_x + \beta_x(\phi \cdot t + \kappa_t) + \gamma_k + \epsilon_{x,t},
    \label{eq:LCC-rewritten}
\end{equation}
where $\mu$ is the intercept. The constraints imposed on the effects are now 
\begin{equation*}
    \sum_x \alpha_x = 0, \quad \sum_x \beta_x = 1, \quad \sum_t \kappa_t = 0,\quad \sum_k \gamma_k = 0. 
\end{equation*}
From this point, we refer to this version of the Lee-Carter model as the LC-model, this version of the cohort-extended Lee-Carter model as the LCC-model. With this reformulation, the effects have the following interpretation:
\begin{itemize}
    \item The intercept $\mu$ can be interpreted as an overall mortality rate
    \item The age effect $\alpha_x$ can be interpreted as the age-specific deviation from the overall mortality rate
    \item The period effect $\kappa_t + \phi \cdot t$ can be interpreted as the period-specific change in the mortality rate, where $\phi \cdot t$ describes the trend relative to the period $t = 0$ and $\kappa_t$ describes the irregularity of this trend. 
    \item The age effect $\beta_x$ can be interpreted as the age-specific sensitivity to the period-related change in mortality. 
    \item the cohort effect $\gamma_k$ can be interpreted as the cohort-specific change in the mortality rate, relative to the overall mortality level. 
\end{itemize}
\subsubsection{Using the \inla Framework for Inference with Lee-Carter models}
\newpar We now investigate how to apply the \inla method for Bayesian inference to the Lee-Carter models. We begin by showing that the Lee-Carter models can be written as LGMs. We write the model in Equation \ref{eq:LC-model} as 
\begin{equation}
    \textbf{y}\mid \textbf{x}, \boldsymbol{\theta} \sim \prod_{x,t}\Poisson(E_{x,t}e^{\eta_{x,t}}\mid \boldsymbol{\theta})
\end{equation}
where $\eta_{x,t}$ is given uniquely by $\{\alpha_x, \beta_x, \kappa_t, \phi, \epsilon_{x,t}\}$ as in Equation \ref{eq:LC-model} and $\Vector{\theta}$ is the vector of hyperparameters, $\Vector{\theta} = \{\tau_{\eta}, \tau_{\alpha}, \tau_{\beta}, \tau_{\kappa}, \tau_\epsilon\}$. The latent field $\Vector{x}$ can then be represented by 
\begin{equation*}
    \Vector{x} = (\alpha_1,\ldots,\alpha_X,\beta_1,\ldots,\beta_X, \phi, \kappa_1,\ldots, \kappa_T, \epsilon_1, \ldots, \epsilon_N)
\end{equation*}
or equivalently 
\begin{equation}
    \Vector{x} = (\eta_1,\ldots,\eta_N,\alpha_1,\ldots,\alpha_X,\beta_1,\ldots,\beta_X, \phi, \kappa_1,\ldots, \kappa_T)
    \label{eq:LChyperparams}
\end{equation}
when $x\in \{1,\ldots,X\}$, $t\in\{1,\ldots,T\}$ and $N=X\times T$. Furthermore, we can now say that the latent field $\Vector{x}$ follows a normal distribution given $\Vector{\theta}$:
\begin{equation}
        \Vector{x}\mid \Vector{\theta} \sim \Normal(\Vector{\mu}(\Vector{\theta}), \Matrix{Q}^{-1}(\Vector{\theta})),
        \label{eq:XNormalDist}
\end{equation}
where $\Vector{\mu}(\Vector{\theta})$ is the mean and $\Matrix{Q}^{-1}(\Vector{\theta})$ is the precision matrix. We have already assigned gaussian priors for $\beta_x$ and $\epsilon_{x,t}$ (and then implicitly for $\eta_{x,t}$) and since the increments in the random walk priors for $\alpha_x$, $\kappa_t$ and $\gamma_k$ are assumed gaussian, the assumption hold for these effects as well. 
\newline
We see that the first criterion is fulfilled, the Lee-Carter models can be written as an LGMs. We also see from Expressions \ref{eq:LC-model} and \ref{eq:LCC-model-orig} that each $y_i$ only depends on the latent random field $\Vector{x}$ through $\eta_i$, so that the second criterion is also fulfilled. This means that the non-linearity of the predictior $\eta_{x,t}$, as discussed in Section \ref{sec:InlaRestrictions} is the only part of the Lee-Carter models that is not compatible with the \inla method. The methodology of the \inlabru framework (see Section \ref{sec:inlabru}) proposes a solution to this, by linearizing non-linear predictors in order to successfully run \inla. We investigate if applying this method to the Lee-Carter models enables us to use the \inla method to make fast Bayesian inference \textcolor{myDarkGreen}{on the Lee-Carter models. }


\subsection{Indexing of age $x$, period $t$ and cohort $k$}
\textcolor{myDarkGreen}{Also not quite sure where to include this section. }
To identify the different age groups, periods and cohorts that is considered by the models, we apply the following indexing scheme:
\begin{itemize}
    \item The age groups included in the population in question are indexed by $x = 1,\ldots,X$ such that age group $x=1$ correspond to the youngest age and age group $x=X$ corresponds to the oldest age.
    \item The periods for which we consider the population in question are indexed by $t=1,\ldots,T$, where period $t=0$ corresponds to the first period and period $t = T$ refers to the most recent period. 
    \item Since the cohort (birth year) is uniquely defined by the age group and the period, we define the cohort index $k$ using these. Following the approach in \cite{rieblerHeld2010}, which was initially proposed by \cite{Heuer1997}, we define the cohort indices by $k = M(X - x) + t$, $K = M(X - 1) + T$. Here $M$ is the difference in length of the age and period intervals, the age intervals are $M$ times wider than the period intervals. For age and period intervals of equal length, the cohort indices become $k = X - x + t$, $K = X - 1 + T.$
\end{itemize}