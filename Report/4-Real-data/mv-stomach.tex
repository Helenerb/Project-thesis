\newpage
\subsubsection{Stomach cancer data}
\textcolor{myDarkGreen}{Instead of repeating everything you said in the lung cancer analysis, you should say that you have done the analysis in exactly the same way, and that the results were very similar, and that you will only comment on the differences. And then briefly present the plots that you have included. }

We apply the APC2- and LCC-models to the stomach cancer data set and analyse the prediction results in the same manner as for the lung cancer data. Overall, we observe that the two models perform similarly on the stomach cancer data set as for the lung cancer data. The score statistics for the predictions are presented in Tables \ref{tbl:mv-LCC-stomach} and \ref{tbl:mv-APC-stomach}. \textcolor{myDarkGreen}{For the stomach cancer data, we do not experience any convergence issues, as described in Section \ref{sec:mv-lung}, with any of the models.} Again, we see that the models with no shared effects and shared period effects have the lowest MDSS, both for the LCC-models and the APC2-models. Out of these, the models with a shared period have a slightly lower MDSS.  

\begin{table}
    \begin{center}
        \begin{tabular}{l |c c c }
            Model & MSE & MDSS & Contained 95\%-interval\\
            \hline
            All common            & 1.019e-7  & -17.04    & 0.9769 \\
            Common age            &  9.69e-8 & -17.92    & 0.7454 \\
            Common age, cohort    & 8.969e-8 & -16.93    & 0.7546 \\
            Common age, period    & 5.991e-8 & -18.04    & 0.8565 \\
            Common cohort         &  3.544e-8 & -18.13   & 0.8333 \\
            Common period         &  4.962e-8 & \textbf{-18.78}   & 0.875  \\
            Common period, cohort & 3.631e-8 & -18.01    & 0.8426 \\
            No common            &  4.929e-8 & \textbf{-18.71}    & 0.8611 \\
        \end{tabular}
        \caption{Score statistics for the different multivariate LCC models, for the stomach cancer data set. The lowest mean DSS values are marked in bold font. }\label{tbl:mv-LCC-stomach}
    \end{center}
\end{table}

\begin{table}
    \begin{center}
    \begin{tabular}{l |c c c }
        Model & MSE & MDSS & Contained 95\%-interval\\
        \hline
        apc    &2.195e-8 &\textbf{-19.32}    &0.9306 \\
        apC    &5.602e-8 &-18.51    &0.8657 \\
        aPc    &2.535e-8 &\textbf{-19.36}    &0.9120 \\
        aPC    &4.120e-8 &-17.57    &0.8611 \\
        Apc    &1.330e-8 &-18.68    &0.9120 \\
        ApC    &2.969e-7  &-15.15    &0.9861 \\
        APc    &3.634e-8 &-18.47    &0.875  \\
        APC    &8.517e-8 &-15.94    &0.9722 \\
    \end{tabular}
    \caption{Score statistics for the different multivariate APC models, for the stomach cancer data set. The lowest mean DSS values are marked in bold font. }\label{tbl:mv-APC-stomach}
    \end{center}
\end{table}

% \begin{figure}[h!]
%     \centering
%     \begin{subfigure}[b]{.45\linewidth}
%         \includegraphics[width=\linewidth]{real-data/real-data-multivariate/Figures/multivariate-LCC-by-age-stomach.png}
%     \end{subfigure}
%     \begin{subfigure}[b]{.45\linewidth}
%         \includegraphics[width=\linewidth]{real-data/real-data-multivariate/Figures/multivariate-LCC-by-period-stomach.png}
%     \end{subfigure}
%     \caption{The two best LCC models - by age (left) and period (right) for the stomach cancer data set}
%     \label{fig:mv-LCC-stomach}
% \end{figure}

% \begin{figure}[h!]
%     \centering
%     \begin{subfigure}[b]{.45\linewidth}
%         \includegraphics[width=\linewidth]{real-data/real-data-multivariate/Figures/multivariate-APC-by-age-stomach.png}
%     \end{subfigure}
%     \begin{subfigure}[b]{.45\linewidth}
%         \includegraphics[width=\linewidth]{real-data/real-data-multivariate/Figures/multivariate-APC-by-period-stomach.png}
%     \end{subfigure}
%     \caption{The two best APC models - by age (left) and period (right) for the stomach cancer data set}
%     \label{fig:mv-APC-stomach}
% \end{figure}

\newpar The prediction results from these models are displayed in Figure \ref{fig:mv-LCC-APC-stomach}. Again, we observe little difference between the APC2 and the LCC-model, other than seemingly wider confidence bounds for the APC2-model. We note that out of the two, the APC2-model has the lowest MDSS. Both models fit the female mortality rates for all ages, for all years quite well. The male mortality rates are also most of the time accurately predicted by both models. The exception is the 85+ age group for the years 2011-2016, and to some degree the 70-74 age group for the same period. \textcolor{myDarkGreen}{You do not have a good explanation for why yet. } 

\begin{figure}
    \centering
    \begin{subfigure}[b]{.45\linewidth}
        \includegraphics[width=\linewidth]{real-data/real-data-multivariate/Figures/multivariate-comparison-by-age-stomach.png}
    \end{subfigure}
    \begin{subfigure}[b]{.45\linewidth}
        \includegraphics[width=\linewidth]{real-data/real-data-multivariate/Figures/multivariate-comparison-by-period-stomach.png}
    \end{subfigure}
    \caption{Comparison of the best APC model and the best LCC model - by period, for the stomach cancer data set}
    \label{fig:mv-LCC-APC-stomach}
\end{figure}

\newpar Figure \ref{fig:effects-LCC-stomach} presents the estimated random effects from the predictions using the "Common period" and "No common" LCC methods. We observe that the structure of the effects are very similar to the estimated effects for the lung cancer data, and we refer to the analysis in Section \ref{sec:mv-lung} for a more thorough review of these. 

\begin{figure}
    \centering
    \begin{subfigure}[b]{.45\linewidth}
        \includegraphics[width=\linewidth]{real-data/real-data-multivariate/Figures/effects-LCC-common-period-stomach.png}
    \end{subfigure}
    \begin{subfigure}[b]{.45\linewidth}
        \includegraphics[width=\linewidth]{real-data/real-data-multivariate/Figures/effects-LCC-no-common-stomach.png}
    \end{subfigure}
    \caption{Plots of the effects for the two best LCC models - LCC with common period effects (left) and LCC with no common effects (right) for the stomach cancer data. The solid red line indicate the division between observed and predicted periods. }
    \label{fig:effects-LCC-stomach}
\end{figure}

